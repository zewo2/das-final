\documentclass[a4paper,12pt]{article}
\usepackage[utf8]{inputenc}
\usepackage{amsmath}
\usepackage{graphicx}
\usepackage{hyperref}
\usepackage{geometry}
\geometry{left=3cm, right=2.5cm, top=2.5cm, bottom=2.5cm}

\title{\textbf{Projeto Final Gitflow}}
\author{Aníbal Freire}
\date{\today}

\begin{document}

% Capa com imagem
\begin{titlepage}
    \centering
    \includegraphics[width=0.5\textwidth]{istec.jpeg} % Insira o caminho da imagem aqui
    \vspace{1cm}

    {\LARGE \textbf{ISTEC - Instituto Superior de Tecnologias Avançadas do Porto} \par}
    \vspace{1.5cm}

    {\Large \textbf{Projeto Final Gitflow} \par}
    \vspace{0.5cm}

    {\large DAS - Desenvolvimento Ágil de Software \par}
    \vspace{5cm}

    \textbf{Aníbal Freire} \\
    Curso: Desenvolvimento de Software \\

    \vfill

    {\large \today\par}
\end{titlepage}

\newpage
\tableofcontents
\newpage

% Conteúdo do relatório
\section{Título do Projeto}
Projeto Final Gitflow.

\section{Introdução}

O objetivo deste projeto é criar um repositório no GitHub, usando o GitFlow, no qual devemos implementar boas práticas de controle de versão e gestão de branches.

\section{Membros da Equipa e Tarefas}

Aníbal Freire
\begin{itemize}
    \item Criação do Github
    \item AAAAAAAAAAAAAAAAAAAAAAAAAA
\end{itemize}
\vspace{0.2cm}

\noindent
Diogo Lima
\begin{itemize}
    \item AAAAAAAAAAAAAAAAAAAAA
\end{itemize}
\vspace{0.2cm}

\newpage
\section{Link para o repositório}

\url{https://github.com/zewo2/das-final}

\section{Código git usado}

a

\section{Ferramentas e Tecnologias}
Para realizar este projeto usufrui das seguintes ferramentas:
\begin{itemize}
    \item Overleaf - Para escrever este relatório
    \item Git - aaaa
    \item MinGW - aaaa
    \item Github - Para  controlar a versão final e certificar que esta estava sincronizada através de todos os dispositivos e permitir a colaboração remota com o resto da equipa.
    \item GIMP - Para editar algumas imagens usadas no relatório
\end{itemize}

\subsection{Overleaf}
    \includegraphics[width=0.2\textwidth]{overleaf.jpg} % Trocar imagem com logo do Overleaf
    \vspace{0.5cm}

O Overleaf [1] é um editor de Latex que não requer qualquer instalação, por ser online, cujo permite a fácil colaboração entre vários membros de um projeto.

Este foi projetado para tornar o processo de escrita e criação de relatórios técnicos mais fácil e rápido, através da utilização de templates e dos seus intuitivos editores Visual e de Código, permitindo utilizadores com diferentes competências realizarem o seu projeto de forma intuitiva.

O Overleaf permite ainda a utilização de gestores bibliográficos para a inserção de citações e referências bibliográficas, sendo alguns exemplos deste tipo de ferramentas o Zotero e o Mendeley.

No âmbito deste projeto, utilizei o Overleaf para me permitir escrever este relatório.

\subsection{Git}
    \includegraphics[width=0.25\textwidth]{git.png} % Trocar imagem com logo do Git
    \vspace{0.5cm}

Git [2]

No âmbito deste projeto, utilizamos o VSCode para escrever, editar e subsequentemente fazer debug e realizar testes no código escrito.

\subsection{MinGW}
    \includegraphics[width=0.25\textwidth]{mingw.png} % Trocar imagem com logo do MinGW
    \vspace{0.5cm}

MinGW [3]

No âmbito deste projeto, utilizamos o VSCode para escrever, editar e subsequentemente fazer debug e realizar testes no código escrito.

\subsection{Github}
    \includegraphics[width=0.25\textwidth]{github.png} % Trocar imagem com logo do github
    \vspace{0.5cm}

O GitHub [4] é uma plataforma de hospedagem de código-fonte e arquivos com controlo de versão utilizado o Git. Ele permite que programadores, utilitários ou qualquer utilizador registado na plataforma contribuam em projetos privados e/ou Open Source de qualquer lugar do mundo. 

Este é amplamente utilizado por programadores para a divulgação dos seus trabalhos ou para que outros programadores contribuam para o seu projeto, para além de promover fácil comunicação através de recursos que relatam problemas ou misturam repositórios remotos (issues, pull request).

O GitHub é mundialmente usado e chega a ter mais de 36 milhões de utilizadores ativos mundialmente contribuindo em projetos comerciais ou pessoais. Hoje o GitHub abriga mais de 100 milhões de projetos, alguns deles sendo conhecidos mundialmente, como o WordPress, GNU/Linux, Atom e Electron.

No âmbito deste projeto, usamos o Github para controlar as versões atuais do projeto, podermos colaborar remotamente e corrigir erros sem haver necessidade de processos complexos e demorados.

\subsection{GIMP}
    \includegraphics[width=0.3\textwidth]{gimp.png} % Trocar imagem com logo do GIMP
    \vspace{0.5cm}

O GIMP (GNU Image Manipulation Program) (originalmente (General Image Manipulation Program) [5], ou traduzido em Português (Programa de Manipulação de Imagem do GNU) é um programa de software "open-source" focado na criação e edição de imagens bitmap, e desenhos vetoriais.

O GIMP foi criado por estudantes como uma alternativa livre e gratuita ao Adobe Photoshop. Foi um projeto universitário que amadureceu bastante e hoje alcança expressiva popularidade, sendo utilizado por artistas amadores e profissionais do ramo.

O GIMP foi criado para a criação ou manipulação de imagens e fotografias, podendo também criar gráficos, logótipos, redimensionar fotos, alterar cores, combinar imagens utilizando camadas, remover partes indesejadas e converter arquivos entre diferentes formatos de imagem digital.

No âmbito deste projeto, utilizei o GIMP para redimensionar e alterar algumas imagens.

\section{Webografia}
\begin{itemize}
    \item [1] Overleaf, "https://overleaf.com/about/features-overview", 2025

    \item [2] Git, "", 2025

    \item [3] MinGW, "", 2025
    
    \item [4] GITHUB, "https://docs.github.com/pt/get-started/
    \newline
    using-git/about-git", 2025
    
    \item [5] GIMP, "https://docs.gimp.org/2.8/pt-BR/introduction.html", 2025
    
\end{itemize}

\section{Considerações Finais}

Conclusão

\end{document}